%-------------------------------------------------------------------------------
%                      Template Naskah Laporan
%               	Berdasarkan format FMIPA Fisika ITB
% 						(c) @andi_syamsul 2023
%-------------------------------------------------------------------------------

%Template.
\documentclass{laporanITB}

%daftar gambar dan tabel
\usepackage[titles]{tocloft}
\renewcommand\cftfigpresnum{Gambar\  }
\renewcommand\cfttabpresnum{Tabel\   }

%hyperlink dan table of content
\usepackage{hyperref}
\newlength{\mylenf}
\settowidth{\mylenf}{\cftfigpresnum}
\setlength{\cftfignumwidth}{\dimexpr\mylenf+2em}
\setlength{\cfttabnumwidth}{\dimexpr\mylenf+2em}

%Bold Face pada Keterangan Gambar
%\usepackage[labelfont=bf]{caption}

%caption dan subcaption
\usepackage{caption}
\usepackage{subcaption}

%-----------------------------------------------------------------
%data laporan
%-----------------------------------------------------------------
\titleind{Template Laporan ITB DENGAN MENGGUNAKAN \emph{TYPESETTING} \LaTeX}

\fullname{ANDI MUHAMMAD NUR FITRAH SYAMSUL}
\idnum{20222013}
\fullnametwo{A-Kun}
\idnumtwo{1234}

%\approvaldate{20 Januari 2023}

\degree{Magister Fisika}

\yearsubmit{2023}

\program{Fisika}

\dept{fisika}


\begin{document}

\cover

%-----------------------------------------------------------------
%muka laporan
%-----------------------------------------------------------------
\tableofcontents
\addcontentsline{toc}{chapter}{DAFTAR ISI}
\listoftables
\addcontentsline{toc}{chapter}{DAFTAR TABEL}
\listoffigures
\addcontentsline{toc}{chapter}{DAFTAR GAMBAR}

%-----------------------------------------------------------------
%Daftar Singkatan [Optional]
%-----------------------------------------------------------------
\singkatan
\noindent

\begin{tabular}{p{20pt}p{3pt}l}
\textbf{I}\\
ITB & & Institut Teknologi Bandung\\
\\
\end{tabular}

\begin{tabular}{p{20pt}p{3pt}l}
\textbf{J}\\
JTA & & Jatinangor\\
\\
\end{tabular}

%-----------------------------------------------------------------
%abstak
%-----------------------------------------------------------------
\begin{abstractind}
Lorem ipsum dolor sit amet, consectetur adipisicing elit, sed do eiusmod tempor incididunt ut labore et dolore magna aliqua. Ut enim ad minim veniam, quis nostrud exercitation ullamco laboris nisi ut aliquip ex ea commodo consequat. Duis aute irure dolor in reprehenderit in voluptate velit esse cillum dolore eu fugiat nulla pariatur. Excepteur sint occaecat cupidatat non proident, sunt in culpa qui officia deserunt mollit anim id est laborum.

Sed ut perspiciatis unde omnis iste natus error sit voluptatem accusantium doloremque laudantium, totam rem aperiam, eaque ipsa quae ab illo inventore veritatis et quasi architecto beatae vitae dicta sunt explicabo. Nemo enim ipsam voluptatem quia voluptas sit aspernatur aut odit aut fugit, sed quia consequuntur magni dolores eos qui ratione voluptatem sequi nesciunt.


\bigskip
\noindent
\textbf{Kata kunci :} \emph{kata kunci}, kata kunci.
\end{abstractind}

\begin{abstracteng}
\emph{
Lorem ipsum dolor sit amet, consectetur adipisicing elit, sed do eiusmod tempor incididunt ut labore et dolore magna aliqua. Ut enim ad minim veniam, quis nostrud exercitation ullamco laboris nisi ut aliquip ex ea commodo consequat. Duis aute irure dolor in reprehenderit in voluptate velit esse cillum dolore eu fugiat nulla pariatur. Excepteur sint occaecat cupidatat non proident, sunt in culpa qui officia deserunt mollit anim id est laborum.}

\emph{Sed ut perspiciatis unde omnis iste natus error sit voluptatem accusantium doloremque laudantium, totam rem aperiam, eaque ipsa quae ab illo inventore veritatis et quasi architecto beatae vitae dicta sunt explicabo. Nemo enim ipsam voluptatem quia voluptas sit aspernatur aut odit aut fugit, sed quia consequuntur magni dolores eos qui ratione voluptatem sequi nesciunt.}

\bigskip
\noindent
\textbf{\emph{Keywords :}} \emph{keyword, keyword}.
\end{abstracteng}

%-----------------------------------------------------------------
%Bab
%-----------------------------------------------------------------
\include{bab1}

\include{bab2}

\include{bab3}

\include{bab4}

\include{bab5}

%-----------------------------------------------------------------
% Daftar Pustaka
%-----------------------------------------------------------------
\bibliography{IEEEabrv,daftar-pustaka}
\addcontentsline{toc}{chapter}{DAFTAR PUSTAKA}

\end{document}